\section{Samenwerken}

\subsection{Remotes}
\begin{frame}{Remotes}
	Remote: andere (volledige) git repo met gedeeld punt in geschiedenis
	\begin{itemize}
		\item \texttt{git remote -v}: toon remotes met url
		\item \texttt{git remote add <naam> <url>}: \texttt{naam} wordt alias voor \texttt{url}
	\end{itemize}
	Goede plekken om je repository te hosten:
	\begin{itemize}
		\item Github: veel gebruikt, Travis, gratis student account (private repo's beperkt)
		\item Bitbucket: alternatief voor Github, onbeperkt private repo's
		\item UU GitLab: intern voor de UU, mailinglijst, onbeperkt private repo's
	\end{itemize}
\end{frame}

\subsection{Bestaande repo overnemen}
\begin{frame}{git clone}
	Kopieer repository en maak remote `origin'
	\begin{itemize}
		\item \texttt{git clone <url>}
		\item \texttt{git clone <url> <map>}
	\end{itemize}
	Als niet je eigen repo en wel toevoegingen maken: forken
\end{frame}

\subsection{Wijzigingen binnenhalen}
\begin{frame}{git fetch}
	\begin{itemize}
		\item Kopieer commits van een remote naar je repo
		\item Je krijgt ze nog niet direct zichtbaar, maar als remote branch:
		\item \texttt{git diff origin/branch} toont wat nieuw is
		\item \texttt{git merge origin/branch} neemt de (wijzigingen in de) commits over
		\item \texttt{git pull} : \texttt{fetch} + \texttt{merge} ineen
		\item Zorg voor een schone working directory!
	\end{itemize}
\end{frame}

\subsection{Wijzigingen publiceren}
\begin{frame}{git push}
	\begin{itemize}
		\item Kopieer commits van je repo naar een remote
		\item Vergelijkbaar met een pull vanaf je remote
		\item Wanneer nieuwe branch: \texttt{git push --set-upstream origin branch}
	\end{itemize}
\end{frame}

\subsection{Pull requests}
\begin{frame}{Pull requests}
	\begin{itemize}
		\item Geen echt git-mechanisme
		\item ``Hee, wil jij \texttt{git pull blabla} doen?''
		\item Kom je veel tegen in Github etc.
	\end{itemize}
\end{frame}

\begin{frame}{Overzicht}
	\begin{tabular}{ll}
		\texttt{git remote}		& Beheer servers waar je vandaan/naar pusht\\
		\texttt{git clone}		& Maak een lokale kopie van een repo\\
		\texttt{git fetch}		& Download wijzigingen, maar hou ze apart (te mergen)\\
		\texttt{git pull}		& Fetch en merge gelijk\\
		\texttt{git push}		& Kopieer commits naar remote
	\end{tabular}
	Doen:
	\begin{enumerate}
		\item Ga naar een niet-repo map
		\item Open \url{https://j.mp/gitgud-repo}, maak account, fork
		\item Clone je geforkte repo
		\item Zet je naam of iets anders op de lijst
		\item Commit, push, pull request via de site
	\end{enumerate}
\end{frame}
