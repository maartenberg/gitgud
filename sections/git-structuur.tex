\section{Structuur}
\frame{
	3 plekken waar je gegevens zitten:
	\begin{tabular}{ll}
		 Working directory& pas je aan, normale map\\
		 Staging area& bestanden die je gaat committen\\
		 Repository / \texttt{.git/}&Bevat metadata, objecten
	\end{tabular}

	Git draait vooral om bestanden hiertussen heen en weer doen
}

\frame{
	Geen lijst van wijzigingen, maar kopie\"en (snapshots):
	\begin{center}
		\includegraphics[width=\textwidth]{images/snapshots.png}
	\end{center}
	(Een version komt hier overeen met een commit, die dus wijst naar versch. versies van bestanden)
}

\frame{
	\frametitle{Commit}
	\begin{itemize}
		\item Info over jou: (gebruikers)naam + email
		\item Bericht: (hopelijk) korte samenvatting van wat er veranderde
		\item Verwijzing naar set snapshots van bestanden (objects)
		\item Voorganger (merge: 2, initial: 0)
		\item Unieke id (40 tekens)
	\end{itemize}
	\alert{Later (wanneer gedeeld met externe server) niet meer aan (te) passen!}
}

%\frame{
	%\frametitle{Tag}
	%todo / verplaatsen naar waar tags worden ge\"introduceerd?
%}
