\section{Basisacties}

\subsection{Repository aanmaken}
\begin{frame}[fragile]{Repository maken}
	%\frametitle{Repository aanmaken}
	\begin{enumerate}
		\item Open terminal (windows: \alert{git-bash})
		\item \texttt{cd} naar map die je wil bijhouden
			(of \texttt{mkdir} een nieuwe)
		\item \texttt{git init}
		\item \texttt{git status}
	\end{enumerate}
	Als het goed is zie je nu \alert{niet}:
	\begin{minted}[bgcolor=light-gray]{text}
fatal: Not a git repository or 
	any of the parent directories: .git
	\end{minted}
\end{frame}

\subsection{Bestanden laten bijhouden}
\begin{frame}[fragile]{Je eerste commit}
	\begin{enumerate}
		\item \texttt{git status}
		\item Bekijk 'untracked files'
		\item \texttt{git add bestand1 bestand2 map1}\\ of:
			\texttt{git add .}\\
			(map pakt alle bestanden erin mee, . is huidige map)
		\item \texttt{git commit}, voer bericht in, opslaan en sluiten\\
			(of: \texttt{git commit -m "Eerste commit"})
	\end{enumerate}
	Resultaat: 
	\begin{minted}[bgcolor=light-gray]{text}
[master 1234abc] Eerste commit
x files added
	\end{minted}
\end{frame}

\begin{frame}{Bestanden ontstagen}
	\texttt{git add} is ongedaan te maken:\\
	\texttt{git reset HEAD <bestand>}
\end{frame}

\begin{frame}{Help mijn commit is niet goed}
	\texttt{git commit --amend}:
	\begin{itemize}
		\item Commit message nog aanpassen
		\item Files die niet gestaged waren
	\end{itemize}
	\alert{Commit moet nieuwste en niet gedeeld zijn}
\end{frame}

\subsubsection{Commit messages}
\begin{frame}{Commit message}
	Iedere commit heeft een message:
	\begin{itemize}
		\item Subject line
		\item Lege regel
		\item Body
	\end{itemize}
\end{frame}

\begin{frame}{Commit message schrijven}
	Niet verplicht, wel handig/netjes:
	\begin{enumerate}
		\item Scheid subject en body met een lege regel (!)
		\item Subject niet langer dan 50 tekens
		\item Subject beginnen met een hoofdletter
		\item Subject niet eindigen met een punt
		\item Subject imperatief: `Verwijder alles' i.p.v. `Verwijdert alles'
		\item Regels van de body op 72 tekens lang
		\item Zet \emph{wat} en \emph{waarom} in de body, \emph{hoe} gebeurt automatisch
	\end{enumerate}
	% Bron: http://chris.beams.io/posts/git-commit/
\end{frame}

\begin{frame}
	\begin{center}
		\includegraphics[width=\textwidth]{images/areas}
	\end{center}
	\begin{itemize}
		\item \texttt{git add} $\rightarrow$ 'Stage Fixes'
		\item \texttt{git commit} $\rightarrow$ 'Commit'
	\end{itemize}
\end{frame}

\subsubsection{Bestanden negeren}
\begin{frame}[fragile]{Bestanden negeren}
	\begin{itemize}
		\item \texttt{git status} geeft onbekende bestanden altijd aan
		\item Oplossing 1: \texttt{.gitignore}:
	\end{itemize}
	\begin{minted}[bgcolor=light-gray]{text}
# alle bestanden in de map bin
bin/*

# alle .exe
*.exe

# maar wel dinges.exe
!dinges.exe
	\end{minted}

	(\texttt{.gitignore} moet wel gecommit worden)
\end{frame}

\begin{frame}{Bestanden alleen voor jezelf negeren}
	\texttt{.git/info/exclude}:
	\begin{itemize}
		\item Alleen op je eigen kopie van repo
		\item Zelfde syntax als \texttt{.gitignore}
	\end{itemize}
\end{frame}

\subsection{Geschiedenis bekijken}
% git log
\begin{frame}{log}
	\begin{tabular}{ll}
		\texttt{git log}& Bekijk geschiedenis van commits (\texttt{--reverse})\\
		\texttt{git show}& Bekijk veranderingen in commit, standaard nieuwste
	\end{tabular}
\end{frame}

% git diff
\begin{frame}{git diff}
	\begin{tabular}{ll}
		\texttt{git diff}&Toon veranderingen in tracked files (unstaged)\\
		\texttt{git diff --staged}&Toon veranderingen klaargezet voor commit\\
		\texttt{git diff HEAD\^}&Toon veranderingen sinds vorige commit
	\end{tabular}
\end{frame}

% git show
\begin{frame}{git show}
	\begin{tabular}{l l}
		\texttt{git show}&Toon info over vorige commit\\
		\texttt{git show id}&Toon info over commit \texttt{id}\\
	\end{tabular}
\end{frame}

\section{Dingen ongedaan maken}

\subsection{Bestand terugdraaien naar onbewerkt}
\begin{frame}{Bestand terug naar vorige commit}
	\texttt{git checkout -- <bestand>} \\
	\alert{Omdat dit niet gecommit was ben je je wijzigingen kwijt!}
\end{frame}

\subsection{Bestanden terugdraaien naar eerdere versie}
\begin{frame}{Bestand(en) terug naar eerdere versie}
	\begin{itemize}
		\item \texttt{git checkout <hash> <bestand>}: enkele bestanden, staget
		\item \texttt{git checkout HEAD <bestand>}: vorige ongedaan maken
		\item \texttt{git checkout <hash>}: detached HEAD
		\item \texttt{git checkout master}: detached HEAD ongedaan maken
	\end{itemize}
\end{frame}

\subsection{Commits ongedaan maken}
\begin{frame}{git revert}
	\begin{itemize}
		\item \texttt{git revert <hash>}: maak nieuwe omgekeerde commit
		\item Safe: reverts van reverts van reverts kan je reverten
	\end{itemize}
\end{frame}

\begin{frame}{git reset}
	Veilig:
	\begin{itemize}
		\item \texttt{git reset}: unstage alles
		\item \texttt{git reset <bestand>}: unstage bestand
	\end{itemize}
	\alert{Onveilig:}
	\begin{itemize}
		\item \texttt{git reset --hard}: \texttt{checkout --} op hele working directory
		\item \texttt{git reset <commit>}: verwijder alle commits na \texttt{<commit>}, laat WD staan
		\item \texttt{git reset --hard <commit>}: \alert{dat is pech, data weg!}
	\end{itemize}
	\alert{Reset alleen lokale dingen!}
\end{frame}

\begin{frame}{git clean}
	\alert{Verwijdert untracked bestanden!}
	\begin{itemize}
		\item \texttt{git clean -n}: dry run
		\item \texttt{git clean -x}: inclusief genegeerd
		\item \texttt{git clean -d}: untracked mappen
		\item \texttt{git clean --force}: \alert{poef}
	\end{itemize}
	(Meestal beter om een snapshot van je branch te downloaden)
\end{frame}
