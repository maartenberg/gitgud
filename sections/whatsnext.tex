\section[Hoe verder?]{Hoe verder na deze workshop?}
\frame{ Vooral veel zelf doen en oefenen! }

\begin{frame}{Mogelijkheden}
	Git heeft heel veel mogelijkheden voor verschillende use-cases:
	\begin{center}
		\begin{tabular}{ll}
			SSH keys		& Niet steeds je wachtwoord invoeren \\
			\texttt{alias} & Verkort veelgebruikte commando's \\
			\texttt{grep}	& Doorzoek alle bij Git bekende bestanden \\
			\texttt{submodules} & Sluit een ander volledig git-repository in \\
			\texttt{stash} & Sla wijzigingen tijdelijk op\\
			Hooks & Voer scripts uit wanneer je iets doet in Git \\
			Travis CI & Bouw en publiceer iets bij elke \texttt{git push}
		\end{tabular}
	\end{center}
	Lees de handleiding: \texttt{git help [commando]}
\end{frame}

\begin{frame}{Extra hulpbronnen}
	\begin{itemize}
		\item Pro Git, het offici\"ele boek: \url{https://git-scm.com/book}
		\item Interactieve tutorials:
			\begin{itemize}
				\item \url{https://try.github.io}
				\item \url{https://codecademy.com/learn/learn-git}
			\end{itemize}
		\item \url{https://www.atlassian.com/git/tutorials/}
		\item Discord: \url{https://discord.gg/qvXvaam} (zie cheatsheet)
		\item \emph{Google} en zelf doen!
	\end{itemize}
\end{frame}

\begin{frame}{Dat was het dan}
	De slides zijn op
\end{frame}
