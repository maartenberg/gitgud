\section{Branching}

\begin{frame}
	\texttt{git status}: On branch master \ldots
\end{frame}

\begin{frame}{Wat is een branch?}
	\begin{itemize}
		\item Branch: verwijzing naar commit (die naar een commit verwijst, die naar \ldots)
		\item Manier om verschillende versies van je geschiedenis te hebben
	\end{itemize}
	Voorbeeld volgt:
\end{frame}

\begin{frame}[fragile]
	\begin{minted}[fontsize=\footnotesize,bgcolor=light-gray]{console}
		$ echo "A" > bestand.txt
		$ git add bestand.txt; git commit -m "Maak bestand A"
		[master (root-commit) fd92806] Voeg bestand toe naar versie A
		 1 file changed, 1 insertion(+)
		  create mode 100644 bestand.txt
		$ git branch volgendeversie
		$ git checkout volgendeversie
		$ git status
		On branch volgendeversie
		nothing to commit, working directory clean
		$ echo "Waardevolle toevoeging die" >> bestand.txt
		$ git add bestand.txt; git commit -m "Begin volgende versie"
		[volgendeversie 5aebde9] Begin volgende versie
		 1 file changed, 1 insertion(+)
		$ git checkout master
		$ cat bestand.txt
		A
		$ git checkout volgendeversie
		$ cat bestand.txt
		A
		Waardevolle toevoeging die
	\end{minted}
\end{frame}

\begin{frame}[fragile]
	\begin{minted}[fontsize=\footnotesize,bgcolor=light-gray]{console}
		$ echo "later pas af is gemaakt" >> bestand.txt
		$ git add bestand.txt; git commit -m "Maakt volgende versie af"
		[volgendeversie 829e8d8] Maakt volgende versie af
		 1 file changed, 1 insertion(+)
		$ git checkout master
		$ git merge volgendeversie
		Updating fd92806..829e8d8
		Fast-forward
		 bestand.txt | 2 ++
		  1 file changed, 2 insertions(+)
		$ cat bestand.txt
		A
		Waardevolle toevoeging die
		later pas af is gemaakt
	\end{minted}
\end{frame}

\subsection{Branches starten}
\begin{frame}{Branches starten}
	\begin{itemize}
		\item Standaard zit je op branch \texttt{master}
		\item Splits hiervanaf met \texttt{git branch <naam>}
		\item Splitsen van een eerdere commit:\\
			\texttt{git branch <naam> <hash>}
		%\item Geef aan dat je op die branch verder wil:\\
			%\texttt{git checkout <naam>}\\
			%(dit is anders dan bijv. svn's \texttt{checkout}!
	\end{itemize}
\end{frame}

\begin{frame}{Branches wisselen}
	Na het maken van een branch moet je aangeven dat je op deze branch verder wil:
	\begin{itemize}
		\item \texttt{git checkout <naam>}
		\item \texttt{git checkout -b <naam>}: maak branch en switch direct
	\end{itemize}
	Hierna committen etc.
\end{frame}

\begin{frame}{Branches samenvoegen}
	\begin{enumerate}
		\item Switch naar branch waar je wijziging naartoe moet
		\item \texttt{git merge <naam>}
			\begin{itemize}
				\item Als base branch geen nieuwe commits heeft: fast-forward (commits worden direct overgenomen)
				\item Als wel: merge commit (2 parents), message typen
			\end{itemize}
	\end{enumerate}
\end{frame}

\subsection{Merge conflicts}
\begin{frame}[fragile]{Merge conflicts}
	Als zelfde gebieden in beide bestanden gewijzigd werkt het vorige niet gelijk:
	\begin{minted}[fontsize=\footnotesize,bgcolor=light-gray]{diff}
		<<<<<<< HEAD
		Aangepast na splitsing!
		=======
		Bestand aangepast van basisbranch
		>>>>>>> dinges
	\end{minted}
	\begin{itemize}
		\item Handmatig bewerken (zie \texttt{git status})
		\item \texttt{git mergetool}
	\end{itemize}
	Wanneer opgelost: commit
\end{frame}

\subsection{Branch management}
\begin{frame}{Branch management}
	\begin{itemize}
		\item \texttt{git branch [-v]}: toon alle branches (en laatste commit)
		\item \texttt{git branch --no-merged}: wat is niet gemerged?
		\item \texttt{git branch -d naam}: verwijder (gemergede) branch
		\item \texttt{git log --graph --all --oneline --decorate} Overzicht van geschiedenis:
	\end{itemize}
\end{frame}

\subsection{Branchen en workflows}
\begin{frame}{Workflows bij branchen}
	\alert{Beste optie: topic branches}
	\begin{itemize}
		\item Je kan altijd \emph{gemakkelijk} terug naar voor je begon
		\item Je kan wijzigingen maken die nog niet helemaal werken / af zijn
		\item Base branch kan wijzigen (samenwerken!) zonder dat je in de problemen zit
	\end{itemize}
\end{frame}
