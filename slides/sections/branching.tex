\section{Branching}

\subsection{Wat zijn branches?}
\begin{frame}
	\texttt{git status}: On branch master \ldots
\end{frame}

\begin{frame}{Wat is een branch?}
	\begin{itemize}
		\item Branch: verwijzing naar commit (die naar een commit verwijst, die naar \ldots)
		\item Manier om verschillende versies van je geschiedenis te hebben
	\end{itemize}
	Voorbeeld volgt
\end{frame}

\begin{frame}
	\begin{center}
		\only<1>{
			\includegraphics[width=.7\textwidth]{graphs/branching1.eps}
		}
		\only<2>{
			\includegraphics[width=.7\textwidth]{graphs/branching2.eps}
		}
		\only<3>{
			\includegraphics[width=.7\textwidth]{graphs/branching3.eps}
		}
		\only<4>{
			\includegraphics[width=.7\textwidth]{graphs/branching4.eps}
		}
		\only<5>{
			\includegraphics[width=.7\textwidth]{graphs/branching5.eps}
		}
		\only<6>{
			\includegraphics[width=.7\textwidth]{graphs/branching6.eps}
		}
		\only<7>{
			\includegraphics[width=.7\textwidth]{graphs/branching7.eps}
		}
		\only<8>{
			\includegraphics[width=.7\textwidth]{graphs/branching8.eps}
		}
		\only<9>{
			\includegraphics[width=.7\textwidth]{graphs/branching9.eps}
		}
	\end{center}
	\only<2>{
		\texttt{git branch branch}
	}
	\only<3>{
		\texttt{git checkout branch}
	}
	\only<4>{
		\ldots, \texttt{git commit}
	}
	\only<5>{
		\texttt{git checkout master}
	}
	\only<6>{
		\texttt{git checkout branch}, \ldots, \texttt{git commit}
	}
	\only<7>{
		\texttt{git checkout master; git merge branch}
	}
	\only<8>{
		\texttt{git checkout master}, \ldots, \texttt{git commit}
	}
	\only<9>{
		\texttt{git merge branch}
	}
\end{frame}

\subsection{Branches starten}
\begin{frame}{Branches starten}
	\begin{itemize}
		\item Standaard zit je op branch \texttt{master}
		\item Splits hiervanaf met \texttt{git branch <naam>}
		\item Splitsen van een eerdere commit:\\
			\texttt{git branch <naam> <hash>}
	\end{itemize}
\end{frame}

\begin{frame}{Branches wisselen}
	Na het maken van een branch moet je aangeven dat je op deze branch verder wil:
	\begin{itemize}
		\item \texttt{git checkout <naam>}
		\item \texttt{git checkout -b <naam>}: maak branch en switch direct
	\end{itemize}
	Hierna committen etc.
\end{frame}

\begin{frame}{Branches samenvoegen}
	\begin{enumerate}
		\item Switch naar branch waar je wijziging naartoe moet
		\item \texttt{git merge <naam>}
			\begin{itemize}
				\item Als base branch geen nieuwe commits heeft: fast-forward (commits worden direct overgenomen)
				\item Als wel: merge commit (2 parents), message typen
			\end{itemize}
	\end{enumerate}
\end{frame}

\subsection{Merge conflicts}
\begin{frame}[fragile]{Merge conflicts}
	Als zelfde gebieden in beide bestanden gewijzigd werkt het vorige niet gelijk:
	\begin{minted}[fontsize=\footnotesize,bgcolor=light-gray]{diff}
<<<<<<< HEAD
Aangepast na splitsing!
=======
Bestand aangepast van basisbranch
>>>>>>> dinges
	\end{minted}
	\begin{itemize}
		\item Handmatig bewerken (zie \texttt{git status})
		\item \texttt{git mergetool}
	\end{itemize}
	Wanneer opgelost: commit
\end{frame}

\subsection{Branch management}
\begin{frame}{Branch management}
	\begin{itemize}
		\item \texttt{git branch [-v]}: toon alle branches (en laatste commit)
		\item \texttt{git branch --no-merged}: wat is niet gemerged?
		\item \texttt{git branch -d naam}: verwijder (gemergede) branch
		\item \texttt{git log --graph --all --oneline --decorate} Overzicht van geschiedenis:
	\end{itemize}
\end{frame}

\subsection{Branchen en workflows}
\begin{frame}{Workflows bij branchen}
	\alert{Beste optie: topic branches}
	\begin{itemize}
		\item Voor iedere feature een nieuwe branch, mergen wanneer klaar
		\item Je kan altijd \emph{gemakkelijk} terug naar voor je begon
		\item Je kan wijzigingen maken die nog niet helemaal werken / af zijn
		\item Base branch kan wijzigen (samenwerken!) zonder dat je in de problemen zit
	\end{itemize}
\end{frame}
